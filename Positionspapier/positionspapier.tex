%%%%%%%%%% Includes %%%%%%%%%%%%%%%%%%%%%%%%%%%%%%%%%%%%%%%%

% Preamble
% ------------------------------------------------------------------------
% **************************  LaTeX - Preambel  **************************
% ------------------------------------------------------------------------
% von: Mike Leneweit
% ========================================================================

% ~~~~~~~~~~~~~~~~~~~~~~~~~~~~~~~~~~~~~~~~~~~~~~~~~~~~~~~~~~~~~~~~~~~~~~~~
% Seitenlayout
% ~~~~~~~~~~~~~~~~~~~~~~~~~~~~~~~~~~~~~~~~~~~~~~~~~~~~~~~~~~~~~~~~~~~~~~~~

\documentclass[a4paper,12pt,openany,parskip,oneside,ngerman,headsepline]{scrbook}
\usepackage{scrhack}

% Seitenränder
\usepackage{geometry}
\geometry{a4paper, top=20mm, left=30mm, right=40mm, bottom=20mm, headsep=10mm, footskip=12mm}


% ~~~~~~~~~~~~~~~~~~~~~~~~~~~~~~~~~~~~~~~~~~~~~~~~~~~~~~~~~~~~~~~~~~~~~~~~
% Sprache & Zeichen
% ~~~~~~~~~~~~~~~~~~~~~~~~~~~~~~~~~~~~~~~~~~~~~~~~~~~~~~~~~~~~~~~~~~~~~~~~


\usepackage{lib/setspace}
\onehalfspacing

% Spracheinstellungen
\usepackage[ngerman]{babel}

% Zeichencodierung
\usepackage[utf8]{inputenc}

% Font
\usepackage[T1]{fontenc}
\renewcommand{\familydefault}{\sfdefault} % Serifenlose Schrift
\usepackage[scaled=.95]{helvet} % Helvetica
%\usepackage{lmodern} % Standardschriftart

% Formatierungen
\usepackage{soulutf8} % Spezielle Formatierungen
\usepackage{textcase} % Text to upper/lower case

% Links
\usepackage[colorlinks=true, linkcolor=black, citecolor=black, breaklinks=true, urlcolor=black]{hyperref}

% ~~~~~~~~~~~~~~~~~~~~~~~~~~~~~~~~~~~~~~~~~~~~~~~~~~~~~~~~~~~~~~~~~~~~~~~~
% Kopf-/Fußzeile
% ~~~~~~~~~~~~~~~~~~~~~~~~~~~~~~~~~~~~~~~~~~~~~~~~~~~~~~~~~~~~~~~~~~~~~~~~

\usepackage{scrpage2}
\pagestyle{scrheadings}
\clearscrplain
\clearscrheadings
\setheadsepline{0pt}
\ofoot[\pagemark]{\pagemark}

% ~~~~~~~~~~~~~~~~~~~~~~~~~~~~~~~~~~~~~~~~~~~~~~~~~~~~~~~~~~~~~~~~~~~~~~~~
% Zitate
% ~~~~~~~~~~~~~~~~~~~~~~~~~~~~~~~~~~~~~~~~~~~~~~~~~~~~~~~~~~~~~~~~~~~~~~~~

% Anführungszeichen
% Zitieren mit \enquote{zitierter Text}
% http://de.wikibooks.org/wiki/LaTeX-W%C3%B6rterbuch:_Anf%C3%BChrungszeichen#Anf.C3.BChrungszeichen_mit_dem_Paket_.22csquotes.22
\usepackage[babel,german=quotes]{csquotes}

% BibLaTeX
% Weiteres unter http://biblatex.dominik-wassenhoven.de/download/DTK-2_2008-biblatex-Teil1.pdf
\usepackage[backend=biber,style=alphabetic,autocite=footnote]{biblatex}

% ~~~~~~~~~~~~~~~~~~~~~~~~~~~~~~~~~~~~~~~~~~~~~~~~~~~~~~~~~~~~~~~~~~~~~~~~
% Sonstiges
% ~~~~~~~~~~~~~~~~~~~~~~~~~~~~~~~~~~~~~~~~~~~~~~~~~~~~~~~~~~~~~~~~~~~~~~~~

% Rechnen
\usepackage{calc}

% Bilder
\usepackage{graphicx}

% Aufzählungen
\usepackage{enumerate} % Aufzählungslabel bei enumerate anpassen (alternativ: \usepackage{enumitem})

% Abkürzungen
\usepackage[printonlyused]{acronym}

% Referenz
\usepackage{nameref}

% Figuren
\usepackage{float} % Stellt die Option [H] fuer Floats zur Verfgung, alternativ [htbp] verwenden

% Fußnoten
\usepackage[multiple]{footmisc}

% ~~~~~~~~~~~~~~~~~~~~~~~~~~~~~~~~~~~~~~~~~~~~~~~~~~~~~~~~~~~~~~~~~~~~~~~~
% Verringerung der BadBoxes
% ~~~~~~~~~~~~~~~~~~~~~~~~~~~~~~~~~~~~~~~~~~~~~~~~~~~~~~~~~~~~~~~~~~~~~~~~

\tolerance 1414
\hbadness 1414
\emergencystretch 1.5em
\hfuzz 0.3pt
\widowpenalty=10000
\vfuzz \hfuzz
\raggedbottom

\addtokomafont{caption}{\small}
\addtokomafont{captionlabel}{\small \bfseries}

% Zusätzliche Befehle
% If-Funktionen
% http://www.matthiaspospiech.de/latex/vorlagen/allgemein/preambel/befehle/
\usepackage{ifthen}

% Testen, ob ein Befehl definiert wurde
\providecommand{\IfDefined}[2]{%
	\ifcsname #1\endcsname
		#2%
	\else
		% do nothing
	\fi
}

% Testen, ob ein Befehl definiert wurde, mit else-Anweisung
\providecommand{\IfDefinedE}[3]{%
	\ifcsname #1\endcsname
		#2%
	\else
		#3%
	\fi
}

\newcommand{\mvc}{Model-View-Controller }

% Bibliographie
\bibliography{bib/quellen}



%%%%%%%%%% Angaben für die Titelseite %%%%%%%%%%%%%%%%%%%%%%%%%%%%%%%%%%%%%%%%

% Titel der Arbeit
\newcommand{\titel}			{Übertragung der Entwurfsvorschrift III auf Lua}
% Name des Autors
\newcommand{\autor}			{Michael Bannas, Pascal Görgen, Marc Reineking, Timo Kluge, Matthias Sobek \& Jan Werder}
% Zenturie des Autors
\newcommand{\zenturie}		{I12a}
% Typ der Arbeit
\newcommand{\typ}			{Positionspapier}
% Studiengang/Fachrichtung
\newcommand{\fachrichtung}	{Wirtschaftsinformatik}
% Bearbeitungszeitraum
\newcommand{\zeitraum}		{20.05.2014}
% Gutachter
\newcommand{\gutachter}		{Johannes Brauer}

%%%%%%%%%% Dokument Beginn %%%%%%%%%%%%%%%%%%%%%%%%%%%%%%%%%%%%%%%%

% - Titelseite/Deckblatt
% - Inhaltsverzeichnis
% - Abbildungsverzeichnis (ggf.)
% - Tabellenverzeichnis (ggf.)
% - Abkürzungsverzeichnis (ggf.)
% - Formelverzeichnis (ggf.)
% - Hauptteil
% - Quellen-/Literaturverzeichnis
% - Eidesstattliche Erklärung



\begin{document}

% Titelseite/Deckblatt
\begin{titlepage}
\includegraphics[width={0.9\textwidth}]{img/NAKLogo.png}
\begin{center}
\IfDefined{typ}{
	\large{\MakeTextUppercase{\so{\typ{}}}} \\
}
\vspace*{0.5cm}
\begin{small}
in der Fachrichtung \\
\fachrichtung{} \\
\end{small}
\vspace*{1cm}
\large{\MakeTextUppercase{\so{Thema}}} \\
\vspace*{1cm}
%\end{center} % ggf.
\IfDefined{titel}{
	\huge{\textbf{\titel{}}} \\
}
%\begin{center} % ggf.
\vspace*{2cm}
\begin{small}
\begin{tabular}{p{5cm}p{6cm}}
\IfDefined{autor}{
	Eingereicht von:				& \autor{} \\
}
\IfDefined{strasse}{				& \strasse{} \\ }
\IfDefined{ort}{					& \IfDefined{plz}{ \plz{} } \ort{} \\ }
\IfDefined{tel}{
	Tel.:						& \tel \\
}
\IfDefined{mail}{
	E-Mail:						& \mail \\
}
\IfDefined{matnr}{
	Matrikelnummer:				& \matnr \\
}
\IfDefined{zenturie}{
	Zenturie:					& \zenturie \\\\
}
\IfDefined{zeitraum}{
	Bearbeitungszeitrum:			& \zeitraum \\\\
}
\IfDefined{gutachter}{
	Gutachter:					& \gutachter \\
}
\IfDefined{zweitgutachter}{
	Zweitgutachter:					& \zweitgutachter  \\
}
\IfDefined{betgutachter}{
	Betrieblicher Gutachter:		& \betgutachter \\
}
\end{tabular}
\end{small} 
\end{center}
\end{titlepage}

\frontmatter

% Römische Seitennummerierung
\pagenumbering{Roman}
% beginnend bei 2
\setcounter{page}{2}
% Inhaltsverzeichnis
\tableofcontents

% Abbildungsverzeichnis
\listoffigures
\addcontentsline{toc}{chapter}{Abbildungsverzeichnis}

% Tabellenverzeichnis
 \listoftables
 \addcontentsline{toc}{chapter}{Tabellenverzeichnis}
 
% Abkürzungsverzeichnis
\chapter{Abkürzungsverzeichnis}


%%%%%%%%%% Hauptteil %%%%%%%%%%%%%%%%%%%%%%%%%%%%%%%%%%%%%%%%
\mainmatter

\chapter{Einleitung}

Als Semesterarbeit für die Vorlesung Abstraktionskonzepte der Informatik soll die Entwurfsvorschrift III in eine andere Programmiersprache übertragen werden. Die Autoren dieses Positionspapiers haben für diese Aufgabe die Sprache Lua gewählt.

\section{Lua}
Lua ist eine imperative und erweiterbare Skriptsprache zum Einbinden in Programme, um diese leichter weiterentwickeln und warten zu können. Lua ist freie Software und steht aktuell unter der MIT-Lizenz. Lua-Programme sind meist plattformunabhängig und werden vor der Ausführung in Bytecode übersetzt. 


\section{Entwurfsvorschrift III}
Die Entwurfsvorschrift III fügt den bisherigen Entwurfsvorschriften die Rekursion hinzu. Damit ist es fortan möglich beliebig verschachtelte und große Datenstrukturen zu verarbeiten. Dafür ist im minimalen Fall ein Selbstaufruf und eine Terminierungsbedingung nötig.

\chapter{Implementierung}
lua implementierungsstuff

\chapter{Fazit}

fazit stuff

% Quellenverzeichnis
\printbibliography
\addcontentsline{toc}{chapter}{Literatur}
\end{document}
